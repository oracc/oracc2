% This file is part of the Digital Assyriologist.  Copyright
% (c) Steve Tinney, 1994, 1995.  It is distributed under the
% Gnu General Public License as specified in /da/doc/COPYING.
%
% $Id: utility.tex,v 0.10 1997/05/16 15:01:41 s Exp s $

% A miscellany of macros that do useful things, both for users and for
% other macros. They're not in any particular order, but they are fairly
% modular.

% 0) Oddments (various origins)
% ===========================
%
% Little macros that encapsulate some oft-used functionality
%

% useful debugging aid: print a message before and after executing a
% macro
\def\beforeandafter#1{\wstdout{Before: \string#1}#1\wstdout{After: \string#1}}

% set things up so you can say \if\defined\... and \define...
\def\defined{defined}     \def\undefd{undefined}
% these need an expandafter to behave correctly with constructs like
% \define{\csname pig\endcsname}
\def\define#1{\global\expandafter\let#1\defined}
\def\undefine#1{\global\expandafter\let#1\undefined}

% test argument #1 to see if it's defined and execute true text (#2) if
% so, false text (#3) otherwise
% e.g. \ifdefined\pig{\message{pig OK}}{\message{no pig}}
\def\ifdefined#1#2#3{\ifx#1\neverdefined#3\else#2\fi}

% like the above, but for use with csname 
% e.g. \ifcsname{pig}{\message{pig OK}}{\message{no pig}}
%%
%% We used to have:
%%
%% \def\ifcsname#1#2#3{\expandafter\if\csname#1\endcsname\relax#3\else#2\fi}
%%
%% but this def is not good enough: expansions of csname which result
%% in two tokens break it (like `11').
%%
%% The following is more complicated, but gets the \relax as first comparand,
%% which is then more robust.
\def\ifcsname#1#2#3{%
  \expandafter\if
    \expandafter\expandafter
      \relax
      \csname#1\endcsname
        #3\else#2\fi}

% Eat arguments so you can ignore them
\def\gobbleone#1{\relax}
\def\gobbletwo#1#2{\relax}
\def\gobblethree#1#2#3{\relax}

% put #1 in math mode if not already in it
\def\mathnomath#1#2{\ifmmode#1\else#2\fi}
\def\forcemath#1{\mathnomath{#1}{$#1$}}

%\def\forcemath#1{\ifmmode\def\next{#1}\else\def\next{$#1$}\fi\next}

% a depth independent, non-math, underline macro
\def\ul#1{\leavevmode\vtop{\hbox{#1\strut}\hrule}}

% remove leading and trailing whitespace from #1
\def\stripwhite#1{\ignorespaces#1\unskip}

% so we can say \def\next\true ... \ifx\next\true
\def\true{t} \def\false{f}

% sheer laziness
\def\glet{\global\let }

% set a piece of text to occupy at least #1 much room, but to run into
% following text if wider than that.
\newskip\posthboxtominamount \newif\ifhboxtominwaslong
\def\hboxtomin#1#2{\setbox0=\hbox{#2}%
  \ifdim\wd0>#1\relax\global\hboxtominwaslongtrue\unhbox0 \unskip
               \hskip\posthboxtominamount\ignorespaces
  \else\global\hboxtominwaslongfalse\hbox to#1{\unhbox0\hfil}\fi}

% put argument at right hand end of line if it fits, or at end of
% next line if not
\def\shoveright#1{{\unskip\nobreak\hfil\penalty50
  \hskip2em\hbox{}\nobreak\hfill #1%
  \parfillskip=0pt \finalhyphendemerits=0 \endgraf}}
% Define \( (or `@(') as a shorthand for \shoveright: this definition
% requires the user to have (...) around the text
\def\(#1){\shoveright{\rm(#1)}}

% Put a box around the argument; the box has default rule width
% (0.04em in Plain TeX).  These macros are based on the TeXbook,
% p.331 ad 21.3.
%
% The documentation should mention that ordinary text should 
% probably be enclosed in an \hbox, or you get a box around a 
% line (para).
\long\def\boxit#1{\vbox{\hrule\hbox{\vrule\kern3pt 
  \vbox{\kern3pt#1\kern3pt}\kern3pt\vrule}\hrule}}

% Like \boxit, but the first argument is the thickness of the rules 
% to use.
\long\def\boxittwo#1#2{\vbox{\hrule height#1\hbox{\vrule width#1\kern3pt 
  \vbox{\kern3pt#2\kern3pt}\kern3pt\vrule width#1}\hrule height#1}}

% Like \boxittwo, but the second argument is the amount of space to clear
% around the box
\long\def\boxitthree#1#2#3{\vbox{\hrule height#1\hbox{\vrule width#1\kern#2
  \vbox{\kern#2\relax#3\kern#2}\kern#2\vrule width#1}\hrule height#1}}

% useful paragraph shapes
\def\centerpar#1\par{{\let\\\break\leftskip=2em plus1fill 
  \rightskip=\leftskip\relax \parindent0pt \parfillskip0pt 
  \leavevmode#1\endgraf}}
\def\leftpar#1\par{\noindent{\leftskip0pt \rightskip=2em plus1fil
   \ignorespaces#1\unskip\endgraf}}
\def\rightpar#1\par{\par{\parfillskip0pt 
  \advance\leftskip0pt plus1fill\relax \noindent
  \ignorespaces#1\unskip\endgraf}}
\def\hangpar#1\par{\par\noindent\hangindent\indention#1\endgraf}

% After the following definitions we can easily apply something to the
% next para, or the next N para's (using \loop). In div.tex this is used
% to achieve the effect of getting an unindented para after the section
% heading without the need to use \noindent (\noindent contributes unwanted
% \parskip glue when \heading is followed immediately by \subheading).
%
% This feature has to be used with some care.  It is intended only for
% controlling the indentation of following paras: lists should use 
% \everypar directly with grouping to control their effects.  Macros 
% that modify \everypar directly should leave it set to \the\nextpar
% after they're done, and may need to call \the\nextpar anyway to 
% ensure that their first para behaves correctly after a heading.
\newtoks\nextpar \everypar{\the\nextpar}

% convenient line types
\def\blankline{\ifvmode\allowbreak\kern\baselineskip\allowbreak
  \else\hfill\break\strut\hfill\break\fi}
\newdimen\rulespace \newdimen\interrulespace 
\rulespace7pt \interrulespace3pt
\def\singlerule{\par\kern\rulespace\hrule\kern\rulespace}
\def\doublerule{\par\kern\rulespace\hrule\kern\interrulespace\hrule%
  \kern\rulespace}

% control appearance of overfull rules
\def\blackboxes{\global\overfullrule5pt}
\def\noblackboxes{\global\overfullrule0pt}

% control sloppiness of paragraphs
\def\verysloppy{\tolerance10000\relax}
\def\sloppy{\tolerance2000\relax}
\def\fussy{\tolerance500\relax}
\def\veryfussy{\tolerance100\relax}

% convenient definitions for things that look better when smaller
\def\@smallrom#1{{\the\scriptfont\the\fam#1}}
\def\@srstart{\unskip~}
\def\I{\@srstart\@smallrom{I}}
\def\II{\@srstart\@smallrom{II}}
\def\III{\@srstart\@smallrom{III}}
\def\IV{\@srstart\@smallrom{IV}}
\def\V{\@srstart\@smallrom{V}}
\def\VI{\@srstart\@smallrom{VI}}
\def\VII{\@srstart\@smallrom{VII}}
\def\VIII{\@srstart\@smallrom{VIII}}
\def\AD{\@srstart\@smallrom{A}.\@smallrom{D}}
\def\CE{\@srstart\@smallrom{C}.\@smallrom{E}}
\def\BC{\@srstart\@smallrom{B}.\@smallrom{C}}
\def\BCE{\@srstart\@smallrom{B}.\@smallrom{C}.\@smallrom{E}}

% shorthands for some things that are a pain to type otherwise
\def\wn{w.n.~\ignorespaces} 
\def\cf{cf.~\ignorespaces} \def\cp{cp.~\ignorespaces} 
\def\eg{\hbox{e.g.}} \def\ie{\hbox{i.e.}} 
\def\sv{s.v.~\ignorespaces} \def\adloc{{\it ad loc}}

% Logos: should add AMS-tex, make one for DA-tex, add MetaFont
\def\TeX{T\kern-.1667em\lower.5ex\hbox{E}\kern-.125emX}
\def\LaTeX{La\TeX{}}

% control puntuation settings
\def\frenchspacing{\sfcode`\.\@m \sfcode`\?\@m \sfcode`\!\@m
  \sfcode`\:\@m \sfcode`\;\@m \sfcode`\,\@m}
\def\nonfrenchspacing{\sfcode`\.3000\sfcode`\?3000\sfcode`\!3000%
  \sfcode`\:2000\sfcode`\;1500\sfcode`\,1250 }

% 1) Ellipsis macros (SJT)
% ===========================
%
% Dots are defined such that ... turns into ellipsis, .... turns into
% an endellipsis and .. issues a warning and outputs '...'.
%
\newif\if@llspace \@llspacefalse \newcount\d@tct \d@tct=0
\def\Pd{.} % this has to be used in <dimens> unless \disabledots is in effect
%% use this one in \macrofile's to prevent dots being switched on again
%\newif\ifdodots \dodotstrue
% macros executed for 1, 2, 3 and 4 dots
%\def\errellipsis{\message{2 dots converted to ellipsis!}\ellipsis}
% This macro should really determine if the next token is a punctuation
% mark and, if so, put in a \nobreak rather than a \penalty2000
\def\@llipstail{\penalty2000\ignorespaces}
\def\ellipsis{\ifvmode\leavevmode\fi\hbox{.\ .\ .}\@llipstail}
\def\endellipsis{\unskip\nobreak\hbox{.\ .\ .\ .}\@llipstail}
%\def\nd@ts{\unskip\loop~.\ifnum\d@tct>0 \advance\d@tct-1 \repeat\ \ignorespaces}
% These macros do the business. For each dot, see if the next token is a
% dot. If so, advance \d@tct and check the next token. When you find a token
% that isn't a dot, or if you've found 4 dots (\d@tct=3), call \out@llips.
%{\catcode`.=\active \gdef.{\chk@llips} \gdef\activedot{\chk@llips}}
%\def\chk@llips{\futurelet\nexttok\tst@llips}
%\def\tst@llips{\ifx\activedot\nexttok\def\nextcs{\adv@llips}%
%  \else\def\nextcs{\out@llips}\fi\nextcs}
%\def\adv@llips#1{\advance\d@tct1\relax\chk@llips} % #1 eats the current dot
%\def\out@llips{\ifcase\d@tct\def\nextcs{\Pd}\or\def\nextcs{\errellipsis}%
%  \or\def\nextcs{\ellipsis}\or\def\nextcs{\endellipsis}%
%  \else\def\nextcs{\nd@ts}\fi\nextcs\d@tct0\relax}
% provide convenient way to switch these macros on and off
%\def\enabledots{\makeactive{.}}\def\disabledots{\makeother{.}}
\let\enabledots\relax \let\disabledots\relax

% 2) Inwords (Raymond Chen, submitted to TeXhax in early 1991)
% ===============================================================
%
% Take an argument in digits and print it in words.
%
\def\inwords{\numbersplit{1000000000}\underbillion\underthousand{ billion }}
\def\numbersplit#1#2#3#4#5{%
  \ifnum#5<#1 #2{#5}\else
    {\count@#5\relax \allocationnumber\count@ \divide\count@ #1\relax
      #3\count@#4\multiply\count@ #1\relax
      \advance\allocationnumber-\count@ #2\allocationnumber\relax}%
  \fi}
\def\underbillion{\numbersplit{1000000}\undermillion\underthousand{ million }}
\def\undermillion{\numbersplit\@m\underthousand\underthousand{ thousand }}
\def\underthousand{\numbersplit{100}\underhundred\undertwenty{ hundred }}
% The careful sidestepping involved in \count@=#1 \allocationnumber=\count@
% is to make sure the right thing happens, even if #1=\count@ or
% #1=\allocationnumber.
% We use \allocationnumber as a scratch count variable.  It and \count@
% are always used inside a group, so their original values will be
% restored when the macros finish their job.
\def\underhundred#1{\ifnum#1<20 \undertwenty{#1}\else
  {\count@#1\relax \allocationnumber\count@ \divide\count@ 10
   \ifcase\count@ \or\or twenty\or thirty\or forty\or fifty\or sixty\or
           seventy\or eighty\or ninety\fi
   \multiply\count@ 10
   \advance\allocationnumber by-\count@
   \ifnum\allocationnumber>\z@ -\undertwenty\allocationnumber\fi
  }\fi}
\def\undertwenty#1{\ifcase#1\or one\or two\or three\or four\or five\or
    six\or seven\or eight\or nine\or ten\or eleven\or twelve\or thirteen\or
    fourteen\or fifteen\or sixteen\or seventeen\or eighteen\or nineteen\fi}

% 3) Time and Date (\today from The TeXbook, Appendix E; \inserttime by SJT)
% ==========================================================================
% 
% Print out today as Month DD, YY
%
\def\thismonth{\ifcase\month\or
  January\or February\or March\or April\or May\or June\or
  July\or August\or September\or October\or November\or December\fi}
\def\thismonthshort{\ifcase\month\or
  Jan\or Feb\or Mar\or Apr\or May\or Jun\or
  Jul\or Aug\or Sep\or Oct\or Nov\or Dec\fi}

\def\today{\thismonth\space\number\day, \number\year}

% Insert time in form hh:mm
\newcount\hours		\newcount\minutes
\def\calctime{%
  \hours=\time		\global\divide\hours by60
  \count255=\hours	\multiply\count255 by60
  \minutes=\time	\global\advance\minutes by-\count255\relax}
\def\inserttime{\calctime\number\hours:\ifnum\minutes<10 0\fi\number\minutes}

% give a complete datestamp of form hh:mm on day mon yr
\def\datestamp{\inserttime\space on \number\day\space\thismonthshort\space\number\year}

% 4) Verbatim Macros (after Kent Eschenberg, KEE@PSUARLC, Sep 88 TeXhax)
% ======================================================================
%
% These macros contain several DA-specific constructs. They will need more
% modifications when the FLOD macros are finished.
%
% Entry points: \vb, \vbfile, \vbfileno. Show respectively a short text, 
% a file and a file with line numbering in verbatim style.
%
\newcount\vbnumbering \newbox\vbbox
\def\vbfont{\Fixedwidth\rm\hyphenchar\the\font-1 }
%  Set action to be taken at a quote, end of line, and embedded blank.
{\catcode`\`=13\relax \gdef\vblquote{\def`{\relax\lq}}}
{\catcode`\^^M=13\relax \gdef\vbeol{\def^^M{\leavevmode \endgraf}}%
 \gdef\vbeolsp{\def^^M{ }}}
{\catcode`\ =13\relax \gdef\vbsmallsp{\let \enskip}}
{\catcode`\ =13\relax \gdef\vbmiddlesp{\let \enskip}}
% Adjust character categories of all special characters
\def\vbcatcodes{%
  \let\do\makeother \dospecials \DAdospecials \let\Prime\vbPrime
  \catcode`\{=12\relax \catcode`\}=12\relax \catcode`\@=12\relax
  \catcode`\^^M=13\relax \catcode`\ =13\relax \catcode`\`=13\relax
  \catcode`\^^L=13\relax \vblquote}
% Show a Few Words Verbatim. The user can specify any closing character 
% and should choose one which does not occur in the verbatim text. When
% the endgroup character is found all the prior catcodes will be 
% restored and verbatim mode will terminate naturally
\def\vb#1{\bgroup\catcode`#1=2 \vbfont\vbcatcodes\vbeolsp\vbmiddlesp}
% Show a file verbatim without line numbering
\def\vbfile#1{\def\@fn{#1}\vbfile@\vbdofile}
% Show a file verbatim with line numbering
\def\vbfileno#1{\def\@fn{#1}\vbfile@\everypar{\vbnum@}\vbnumbering0 \vbdofile}
\def\vbnum@{\advance\vbnumbering1 
  \rlap{\hbox to 3em{\hfil\the\vbnumbering\quad}}\hskip3em\relax}
\def\vbfile@{\par\begingroup\verbatimformat\vbeol\vbsmallsp
  \vbfont\ninepoint\baselineskip=2ex \lineskiplimit2pt \lineskip3pt
  \def\vbstrut{\vrule height1.6ex depth.4ex width\z@}%
  \parindent=0in \parskip=0pt \vbhook}
\def\vbdofile{\input\@fn\relax\endgroup\par}
% set up ^^L to act as a pagebreak when active
{\catcode`\^^L\active \let^^L\relax
\gdef\ControlLpages{\catcode`\^^L\active \let^^L\newpage}}
\def\vbhook{\ControlLpages}

% 5) A robust way of shifting formats (SJT)
% =========================================
%
% To extend, simply add another \chardef for the new format, modify the
% \ifcase in \currformat, and define a macro which sets the character
% codes and sets \currfmt, analogous to \DAformat and friends.
%
\chardef\plainfmtc@de		0
\chardef\macrofmtc@de		1
\chardef\DAfmtc@de		2
\chardef\verbatimfmtc@de	3
\newcount\currfmt	\currfmt\macrofmtc@de
\def\currformat{\ifcase\currfmt
  \plainformat%0 
  \or\macroformat%1
  \or\DAformat%2
  \or\verbatimformat%3
  \fi}
\def\effectmacroformat{\catcode92=0 \catcode`\#=6 \catcode95=8 
   \catcode`\^=7 \catcode`\_=10
   \catcode34=12 \catcode64=11 \disabledots}
\def\plainformat{\effectmacroformat\makeother{@}\global\currfmt\plainfmtc@de\catcode`\\0\relax}
\def\macroformat{\effectmacroformat\global\currfmt\macrofmtc@de }
\def\DAformat{\global\currfmt\DAfmtc@de\DAcatcodes}
\def\verbatimformat{\global\currfmt\verbatimfmtc@de\vbcatcodes}

% 6) Files and Filenames (SJT)
% ============================
%
% \checkfile: Check if a file exists and set \ifitexists to true or false
%
%	#1 = the filename
%
\newif\ifitexists
\def\checkfile{\effectmacroformat \ch@ckfile}
\def\ch@ckfile#1{{\immediate\openin0 #1\relax
  \ifeof0\global\itexistsfalse\else
  \global\itexiststrue\fi 
  \immediate\closein0 }\currformat}

% define IfFileExists for xmltex's benefit
\def\IfFileExists#1#2#3{
  \def\@filef@und{#1 }
  \immediate\openin0 #1\relax
  \ifeof0 \def\next{#3}\else\def\next{#2}\fi
  \immediate\closein0 \next}

% \pathfind: find a file somewhere in a path.
%
% The path must be previously defined with the command
% \pathdefine#1#2:
%	#1 = the symbolic path name, without any escape char.
%	#2 = the path: each element must be legal for the local OS, and
%		must include any final delimiter which goes between the
%		directory and the file name. Elements in the path are 
%		delimited by ',', e.g., 
%			e:/bin/,e:/usr/bin/,e:/usr/local/bin/
% For each element in #2 a control sequence \<NAME>n is defined with the
% value of that segment. The command
%	\pathdefine{TFMpath}{e:/emtex/tfms/,d:/da/tex/tfms/}
% is equivalent to
%	\def\TFMpath{e:/emtex/tfms/,d:/da/tex/tfms/}
%	\def\TFMpath1{e:/emtex/tfms/}
%	\def\TFMpath2{d:/da/tex/tfms/}
% except that the digit is part of the \csname, not an argument delimiter.
%
% Thereafter a file may be sought in the defined path by the call
% \pathfind#1#2:
%	#1 = the symbolic path name, as in \pathdefine
%	#2 = the name of the file
%
% The full successful result is returned in the macro \pathresult, which
% is empty on failure.
%
\def\pathdefine{\effectmacroformat\count255=0 \pathdefine@}
\def\pathdefine@#1#2{\advance\count255by1 
  \ifnum\count255=1
    \expandafter\gdef\csname#1\endcsname{#2}%
    \gdef\pathname{#1}\gdef\pathpath{#2}%
    \pathshow{`\pathname' defined as `\pathpath'}%
  \fi
  \pathdefelement#2,\endofarg{#1}{\the\count255 }%
  \nexttry}
\def\pathdefelement#1,#2\endofarg#3#4{%
  \def\next{#1}%
  \ifx\next\empty
    \gdef\nexttry{\currformat}%
  \else
    \expandafter\gdef\csname#3#4\endcsname{#1}%
    \pathshow{`#3#4' defined as `#1'}%
    \gdef\nexttry{\pathdefine@{#3}{#2}}%
  \fi}

\def\pathfind{\effectmacroformat\count255=0 \pathfind@}
\def\pathfind@#1#2{\advance\count255by1 
  \ifnum\count255=1
    \gdef\pathname{#1}\gdef\pathfile{#2}%
%    \pathshow{searching `\pathname' for `\pathfile'}%
  \fi
%%  \gdef\next{\csname#1\the\count255 \endcsname}% %% DOES THIS DO ANYTHING?
  \pathfindelement{#1}{\the\count255 }{#2}\nexttry}
\def\pathfindelement#1#2#3{%
  \gdef\next{\csname#1#2\endcsname}%
  \if\next\empty \relax
    \pathshow{`\pathfile' not found in `\csname\pathname\endcsname'}%
    \gdef\pathresult{}%
    \gdef\nexttry{\currformat}%
  \else
    \checkfile{\next#3}%
    \ifitexists
      \pathshow{found `\next#3'}%
      \xdef\pathresult{\next#3}%
      \gdef\nexttry{\currformat}%
    \else
%      \pathshow{`\next#3' unsuccessful}%
      \gdef\nexttry{\pathfind@{#1}{#3}}\fi
  \fi }

% If \tracingpaths=1 is true we show every path operation on screen and
% log. If \tracingpaths=2 we show operations on log only. (If 0, no show.)
\newcount\tracingpaths
\def\pathshow#1{\ifcase\tracingpaths\or\wstdout{Path: #1}\or\wlog{Path: #1}\fi}

% 7) Protecting Mark and Message (SJT)
% ====================================
%
% These macros provide no protection per se, but they provide the
% programmer with an easy way of getting protection. Just use the
% \@protect macro as an inner shell for any macro that may misbehave
% in a \mark or \message. This may mean that the behaviour in the
% mark/message must be simplified, but there are difficulties with
% TeX's scanning of the token arguments to these commands which
% make it unwise to use \def, \let and conditionals within them.
% Sometimes a trade off is inevitable. 
%
\newif\ifprotecting \newif\ifmessage \newcount\protected@flag
%\let\protected@message\message \let\protected@mark\mark
%\def\message{\messagetrue\call@protected\protected@message}
%\def\mark{\call@protected\protected@mark}
\def\call@protected#1#2{\protectingtrue\def\pn@xt{}%
  #1{#2}%
  \protectingfalse\ifmessage\messagefalse\fi}
\def\mess@protected#1{\ifmessage#1^^J\else\immediate\write16{#1}\fi}
\def\@protect#1#2{\ifprotecting#1\else\def\pn@xt{#2}\fi\pn@xt}

% 8) performing \write's without expansion (TeXbook p. 377)
% =========================================================
%
% As noted ad loc, the second parameter to \write should not contain
% more than about 150 tokens.
%
% Use \rawwrites to enable unexpanded writing; \cookedwrites to enable
% normal expanded writing
%
\let\original@write\write
\long\def\unexpandedwrite#1#2{\def\finwrite{\write#1}%
  {\aftergroup\finwrite\aftergroup{\sanitize#2\endsanity}}}
\def\sanitize{\futurelet\next\sanswitch}
\def\sanswitch{\let\n@xt\endsanity \ifx\next\endsanity
  \else\ifcat\noexpand\next\stoken\aftergroup\space\let\n@xt=\eat
   \else\ifcat\noexpand\next\bgroup\aftergroup{\let\n@xt=\eat
    \else\ifcat\noexpand\next\egroup\aftergroup}\let\n@xt=\eat
     \else\let\n@xt=\copytok\fi\fi\fi\fi \n@xt}
\def\eat{\afterassignment\sanitize \let\next= }
\long\def\copytok#1{\ifcat\noexpand#1\relax\aftergroup\noexpand\fi
  \ifcat\noexpand#1\noexpand~\aftergroup\noexpand\fi
  \aftergroup#1\sanitize}
\def\endsanity\endsanity{}
\def\rawwrites{\let\write\unexpandedwrite}
\def\cookedwrites{\let\write\original@write}

% 9) Miscellaneous Reading and Writing Manoeuvres (SJT)
% =====================================================
%
% provide a safe shell for reading macro files. The control sequence
% \macrofile is the only documented one of the \macrofile/\inputM pair.
% \inputM is retained solely for backward compatibility with previous
% versions of this format. We take some care to be reentrant, but the
% format-switching code really needs to be redone to push/pop formats.
\newif\ifmacrofile \macrofilefalse \newcount\savedfmt
\def\macrofile{\ifmacrofile\def\next{\inp@tm}\else
  \def\next{\macrofiletrue\savedfmt\currfmt\macroformat\inp@tM}\fi
  \next}
\def\inp@tM#1{\input #1 \relax\currfmt\savedfmt\currformat\macrofilefalse} 
\let\inputM\macrofile
\def\inp@tm#1{\input #1 \relax}

% and another for document files
\def\include{\bgroup\effectmacroformat\afterassignment\@nclude\xdef\next}
\def\@nclude{\egroup\input\next\relax}

% redefine \message so it will expand its text in macro format
%\let\m@ss@ge\message
%\def\message{\effectmacroformat\m@ssage}
%\def\m@ssage#1{\m@ss@ge{#1}\currformat}

% define a macro for the everyjob tokens
\def\readlocal{\checkfile{local}%
  \ifitexists\def\next{\inputM{local}}\else\let\next\relax\fi
  \next}

% futurenonspacelet from the TeXbook
\def\futurenonspacelet#1{\def\cs{#1}%
  \afterassignment\stepone\let\nexttoken= }
{\def\\{\glet\stoken= } \\ }
\def\stepone{\expandafter\futurelet\cs\steptwo}
\def\steptwo{\expandafter\ifx\cs\stoken\let\next=\stepthree
  \else\let\next=\nexttoken\fi \next}
\def\stepthree{\afterassignment\stepone\let\next= }

\def\leavehmode{\ifhmode\endgraf\fi}

%%% These don't have a better home at the moment...
\def\bSuper{\setbox0=\hbox\bgroup\scriptsize}
\def\eSuper{\egroup\raise3pt\box0\relax}
\def\bText{\leavevmode\rm}
\gdef\Lsux{sux}
\gdef\Lakk{akk}
\gdef\Lqpc{qpc}
\gdef\logorole{logo}\gdef\signrole{sign}
\def\rolesize{\ifx\role\signrole\else\logosize\fi}%

\endinput

%%%%%%%%%%%%%%%%%%%%%%%%%%%%%%%%%%%%%%%%%%%%%%%%%%%%%%%%%

$Log: utility.tex,v $
Revision 0.10  1997/05/16 15:01:41  s
correct use of arguments in boxittwo

Revision 0.9  1996/05/27 16:37:56  s
change verbatimformat back to vbcatcodes in \vb only (verbatimformat
broke \vb somehow)

% Revision 0.8  1996/05/27  00:40:05  s
% replace \vbcatcodes with \verbatimformat twice
%
% Revision 0.7  1996/02/17  23:33:44  s
% refine definition of \rightpar so that leftskip is increased by fill,
% but no other amount
%
% Revision 0.6  1995/11/22  11:35:36  s
% add cp to useful shorthands
%
% Revision 0.5  1995/11/18  11:22:16  s
% make \hboxtomin use \posthboxtominamount instead of a \quad
% for the space after text which is wider than min
%
% Revision 0.4  1995/07/01  12:21:21  s
% added \boxit and \boxittwo
%
% Revision 0.3  1995/06/16  11:06:45  s
% added \BCE to the `better when smaller' group
%
% Revision 0.2  1995/02/27  14:35:42  s
% added _ to \effectmacroformat
% fixed \include to use macro format for scanning filenames
%
% Revision 0.1  1994/12/12  03:56:05  s
% initial checkin
%
