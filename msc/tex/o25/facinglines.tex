\def\facinglines{}
\def\endfacinglines{}

\newif\if@flushing \newif\ifFLn@break  \newif\ifFLheading

\newdimen\FLcolwidthOne \newdimen\FLcolwidthTwo \newdimen\FLcolwidthThree
\newdimen\FLgutterOne \newdimen\FLgutterTwo \newdimen\FLrulingadjust

\newskip\FLbeforeheadingskip \FLbeforeheadingskip5pt plus1pt minus1pt
\newskip\FLafterheadingskip \FLafterheadingskip4pt plus1pt minus1pt

\def\FLsetcolumns#1#2#3#4#5{%
  \FLcolwidthOne=#1
  \FLgutterOne=#2
  \FLcolwidthTwo=#3
  \FLgutterTwo=#4
  \FLcolwidthThree=#5 \relax
}

%\def\FLnotranscolumns{%
%  \FLcolwidthOne=.6in
%  \FLgutterOne=.1in
%  \FLcolwidthTwo=5.8in
%  \FLcolwidthThree=0in
%  \FLgutterTwo=0in \relax
%}

%\def\FLnotrans{\FLnotranscolumns}

\def\FLblockskip{\vskip9pt\relax\filbreak}

\newdimen\FLsavedvfuzz

\newcount\FLlinecountOne \newcount\FLlinecountTwo \newcount\FLlinecountThree
\newcount\FLmaxLines 
\newcount\FLiterations

\newbox\FLboxOne \newbox\FLboxTwo \newbox\FLboxThree

\newbox\FLtmpOne \newbox\FLtmpTwo \newbox\FLtmpThree

\def\FLforceleft{\leavevmode\kern-\FLcolwidthOne\relax\kern-\FLgutterOne}

\def\FLcountlines#1#2{%
  #1=0\relax
  \FLsavedvfuzz=\vfuzz
  \vfuzz1000pt
  \setbox0=\vbox{\unvcopy#2}%
  \loop
    \setbox2=\vsplit0 to1pt
    \advance#1by1
  \ifdim\ht0>0pt\repeat
  \vfuzz\FLsavedvfuzz\relax
}

\def\FLstrut{\vrule width0pt height 9pt depth1.7pt{}}
\def\FLvstrut{\hrule width0pt height 9pt depth1.7pt{}}

\newskip\zzz
\def\|{\FLstrut\hskip\zzz}

\def\FLbaseline{\baselineskip9.7pt \lineskiplimit0pt \lineskip0pt\relax
  \FLrulingadjust=15pt\relax
}

\def\FLnextline#1#2#3{%
  \ifdim\ht #2 > 0pt \relax
    \FLsavedvfuzz=\vfuzz\relax
    \vfuzz10000pt
    \setbox2=\vsplit#2 to1pt
    \vfuzz=\FLsavedvfuzz
  \else
    \setbox2=\vbox{\hbox to#3{\hfil}\vfil}%
  \fi
  \box2
}

\def\FLheading{\global\FLheadingtrue\bf}

\def\FLpar{\clubpenalty0 \widowpenalty0 \brokenpenalty0
  \interlinepenalty0 \parindent0pt \hangindent0pt \parskip0pt\relax }

\def\FLsuppressbreak{\global\FLn@breaktrue}

\def\FLtlit{\spaceskip3.3333pt \tolerance100 \rightskip0pt plus1fil
            \relax}

\def\FLtlat{\tolerance3000 \pretolerance1000 \hyphenpenalty0 
  \spaceskip3.33333pt plus1.66666pt minus1.11111pt
  \lineskip2pt \lineskiplimit2pt
  \baselineskip12pt \relax}

\newdimen\FLcoltwoindent \newbox\FLcalccoltwoindentbox
\newdimen\FLlnumfuzz \FLlnumfuzz3pt %2.2pt
\def\FLcalccoltwoindent#1{%
  \setbox\FLcalccoltwoindentbox=\hbox{#1}% \kern2pt
  \ifdim\wd\FLcalccoltwoindentbox>\FLcolwidthOne
    \dimen0\wd\FLcalccoltwoindentbox \advance\dimen0-\FLcolwidthOne
    \ifdim\dimen0 > \FLlnumfuzz
      \message{FLcalccoltwoindent #1 is too wide (\the\wd\FLcalccoltwoindentbox > \the\FLcolwidthOne)}%
      \FLcoltwoindent=5pt
    \else
      \FLcoltwoindent=0pt
    \fi
  \else
    \FLcoltwoindent=0pt
  \fi
}

\newbox\lnumbox
\def\lnuminbox#1{\setbox\lnumbox=\hbox{#1}\wd\lnumbox=0pt\box\lnumbox}

% We're going to format args #2 and #3 and then
% pass one line at a time to the page builder,
% with all three columns wrapped in a single hbox.
\def\facingblock#1#2#3#4{%
  \gdef\FLnotes{#4}%
  \let\next\relax
  \ifdim\ht\FLboxThree>0pt
    \let\next\@facinglines
  \else
    \ifdim\ht\FLboxTwo>0pt
      \let\next\@facinglines
    \else
      \ifdim\ht\FLboxOne>0pt
        \let\next\@facinglines
      \fi
    \fi
  \fi
  \@flushingtrue\next
  \@flushingfalse
  \FLcalccoltwoindent{#1}%
  \setbox\FLboxOne=\vbox{\FLpar\hsize\FLcolwidthOne\relax
        \FLbaseline\FLstrut\lnuminbox{#1}\endgraf}%
  \setbox\FLboxTwo=\vbox{\FLpar\FLtlit
        \hsize\FLcolwidthTwo\relax
        \FLbaseline\FLstrut\kern\FLcoltwoindent\relax
        \ifFLheading\bf\fi#2\endgraf}%
  \setbox0=\hbox{#3\hfil}%
  \ifdim\wd0>0pt
    \setbox\FLboxThree=\vbox{\FLpar\FLtlat\hsize\FLcolwidthThree\relax
          \FLbaseline\ifFLheading\bf\fi#3\endgraf}%
  \fi
  \@facinglines
}

\def\facingmoretlit#1#2#3{%
  \@flushingfalse
  \gdef\FLnotes{#3}%
  \FLcalccoltwoindent{#1}%
  \setbox\FLboxOne=\vbox{\FLpar\hsize\FLcolwidthOne\relax
                         \FLbaseline\FLstrut\lnuminbox{#1}\endgraf}%
  \setbox\FLboxTwo=\vbox{\FLpar\FLtlit\hsize\FLcolwidthTwo\relax
                         \FLstrut\kern\FLcoltwoindent\relax#2}%
  \@facinglines
}

\def\facingflush{%
  \let\next\relax
  \@flushingtrue
  \ifdim\ht\FLboxThree>0pt
    \let\next\@facinglines
  \else
    \ifdim\ht\FLboxTwo>0pt
      \let\next\@facinglines
    \else
      \ifdim\ht\FLboxOne>0pt
        \let\next\@facinglines
      \fi
    \fi
  \fi
  \next
}

\def\facingrulings{\unskip\facingflush
  \dimen0\FLcolwidthOne \advance\dimen0\FLgutterOne 
  \advance\dimen0\FLcolwidthTwo
  \nobreak\kern10pt\nobreak\nointerlineskip\nobreak
  \hbox{%
%    \kern\FLcolwidthOne \kern\FLgutterOne
    \vrule width\dimen0 height.5pt
    \kern\FLgutterTwo
    \vrule width\FLcolwidthThree height.5pt
  }\kern-8pt\relax
  \allowbreak
}

\def\@facinglines{%
  \FLcountlines\FLlinecountOne\FLboxOne
  \FLcountlines\FLlinecountTwo\FLboxTwo
  \FLcountlines\FLlinecountThree\FLboxThree
  \FLmaxLines=\FLlinecountOne
  \ifnum\FLlinecountTwo>\FLmaxLines
    \FLmaxLines\FLlinecountTwo
  \fi
  \if@flushing
    \ifnum\FLlinecountThree>\FLmaxLines
      \FLmaxLines\FLlinecountThree
    \fi
  \fi
  \let\next\relax
  \ifdim\lastskip>14pt \def\next{\FLn@breaktrue\nointerlineskip}\fi
  \next
  \ifFLheading\goodbreak\vskip\FLbeforeheadingskip\fi
  \FLiterations\FLmaxLines
  \loop
    \setbox0=\hbox to\fullhsize{%
      \FLnextline\FLtmpOne\FLboxOne\FLcolwidthOne
      \kern\FLgutterOne
      \FLnextline\FLtmpTwo\FLboxTwo\FLcolwidthTwo
      \kern\FLgutterTwo
      \FLnextline\FLtmpThree\FLboxThree\FLcolwidthThree
      \hfill
    }%
    \ifFLn@break
       \global\FLn@breakfalse\nobreak
    \else
      \allowbreak
    \fi
    \hbox{\box0
      \FLnotes\global\let\FLnotes\relax}
    \advance\FLiterations-1
  \ifnum\FLiterations>0\repeat
  \ifFLheading
    \nobreak\vskip\FLafterheadingskip
    \nobreak
    \else\vskip0pt plus1pt minus1pt\fi
  \global\FLheadingfalse
}

\endinput
